\documentclass[tikz, border=10pt]{standalone}
\usepackage{tikz}
\usetikzlibrary{
    positioning,
    arrows.meta,    % <-- This line loads the 'Stealth' arrow tip
    shapes.geometric
}

\begin{document}

\begin{tikzpicture}[
    node distance=2.5cm and 2cm,
    definition/.style={
        rectangle, rounded corners, draw=blue!80, fill=blue!15, thick,
        text width=4cm, text centered, minimum height=1.75cm
    },
    theorem/.style={
        rectangle, rounded corners, draw=green!60!black, fill=green!15, thick,
        text width=4cm, text centered, minimum height=1.75cm
    },
    hypothesis/.style={
        rectangle, rounded corners, draw=red!80, fill=red!15, thick,
        text width=4cm, text centered, minimum height=1.75cm
    },
    arrow/.style={
        -{{Stealth[length=3mm, width=2mm]}}, % Style defined in arrows.meta
        thick,
        draw=black!60
    }
]

% Node Placement
\node[theorem] (op) {odd\_primes\_only};
\node[hypothesis, right=of op] (flt3) {fermatLastTheoremThree};
\node[definition, below=2cm of op, xshift=3cm] (fp) {FreyPackage};
\node[theorem, below=of fp, xshift=-3cm] (notflt) {not\_FLT\_p\_ge\_5};
\node[definition, below=of fp, xshift=3cm] (fc) {FreyCurve};
\node[definition, below=2cm of notflt, xshift=3cm] (fctgr) {FreyCurve\_torsionGaloisRepresentation};
\node[hypothesis, below=of fctgr] (wiles) {Wiles\_Frey};

% Edge Drawing
\path[arrow] (fp.south) edge (notflt.north);
\path[arrow] (fp.south) edge (fc.north);
\path[arrow] (fc.south) edge (fctgr.north);
\path[arrow] (fp.west) edge[bend right=20] (fctgr.north);
\path[arrow] (fctgr.south) edge (wiles.north);
\path[arrow] (fc.south) edge[bend right=20] (wiles.north);
\path[arrow] (fp.south) edge[bend left=45] (wiles.north);

\end{tikzpicture}

\end{document}